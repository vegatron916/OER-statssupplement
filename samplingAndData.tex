\chapter{Sampling \& Data}

\section[Experiments \& Studies]{Identify the differences between experiments and observational studies. Identify the elements of experiments and observational studies including: factors.}

Observational studies and experiments are important surveying methods in collecting data with clear differences that should be examined.  

In an observational study, researchers do not have control over what occurs, they merely identify variables of interest and collect data.  Data may already exist, or it may be gathered as time passes.  A study using existing data is called a \emph{retrospective study}; a study collecting data as time passes is called a \emph{prospective study}.

\begin{example} 
A community college investigated the highest math class taken by any student finishing with a one or two year certificate within the last ten years. To do this, they examined all records and recorded the information. 
Examining existing records makes this a retrospective study.
\end{example}

\begin{example}
A new class is created at a high school and faculty want to collect information about the class, such as grades of students. Since data will be collected as time passes, this is a prospective study.
\end{example}

In an experiment, researchers do have control over what occurs. 

\begin{example}
Consider the simplified description of an experiment 

{http://www.ncbi.nlm.nih.gov/pubmed/18065594}. 

Sixty-six centenarians (people aged 100 or more) were split into two groups by researchers.  Once group received a daily dose of L-Carnitine, the other received a placebo. After the study was complete, the research team used the Mini-Mental State Examination (MMSE) to measure each centenarian's mental fatigue and cognitive function. Results showed a statistically significant increase in capacity for cognitive activity for those centenarians given L-Carnitine compared to those give the placebo. 

An experimental diagram of this situation can be seen in Figure \ref{fig:expDesign} 

\begin{figure}
\centering
\includegraphics[width=4in]{images/expDesign.png}
\caption{Diagram of L-Carnitine experiment.}
\label{fig:expDesign}
\end{figure}

\end{example}

In experiments, researchers look for a relationship between at least two variables. Assuming we only have two variables, we begin by identifying \emph{factors} that we want to manipulate (the dosage received) and \emph{response variables} to measure (mental fatigue or cognitive function). When factors are manipulated, their specific values are known as \emph{levels} (if they receive a dose of L-Carnitine or the placebo). We then find \emph{subjects} to participate in the experiment (people aged 100 or more). Those subjects are then randomly assigned to a combination of levels from all factors, known as a \emph{treatment} (they receive a dose that is a placebo.)


\section[Multistage Sampling]{Identify sample designs including: multistage sample.}

Multistage sampling combines more than one sampling method.  

\begin{example}
During election cycles, polling agencies will do a multistage samples.  As a first stage, they may do an SRS of a large number of election districts, perhaps 1000.  As a second stage, they would sample 10 household within each of those districts.
\end{example}

\begin{example}
The Portland City of Portland wishes to investigate neighborhood street conditions and will use a multistage sample.  As a first stage, they do a cluster sample of Portland neighborhoods.  As a second stage, instead of sampling (or checking) every street in the clusters, they do a random sample of streets and investigate those.
\end{example}

\section[Biased vs.~Unbiased]{Identify and describe terminology: Biased vs.~Unbiased}

In one dictionary, bias is defined as ``prejudice in favor of or against one thing, person, or group compared with another, usually in a way considered to be unfair.'' Wikipedia defines bias to be ``an inclination towards something, or a predisposition, partiality, prejudice, preference, or predilection.'' These definitions match many people's general ideas of the meaning of bias. This kind of bias can be conscious or unconscious, and it may affect people's lives and livelihoods.

In statistics, \emph{bias} is defined as ``the systematic distortion of a statistic'' (Wikipedia). Systematic distortion of a statistic often arises from the sampling process. A statistic from a voluntary response sample, for example, nearly always has bias toward extreme views. 

A biased sample is not representative of the population from which the sample was drawn. That means the statistics of the sample do not reflect the parameters of the population. Good sampling methods avoid bias. An unbiased sample represents its population well in the sense that each sample statistic is a good estimate of the corresponding population parameter. Unfortunately, sources of bias in a sample are not always clear to researchers, but ethical researchers do their best to use a representative sample, minimizing the effect of bias. 

\section{Homework}

\begin{enumerate}
 \item Decide if each of the following situations is a prospective or retrospective study.

 \begin{enumerate}
  \item The City of Portland wants to find out how many one bedroom, two bedroom and three bedroom units there are in the city.
  \item The DMV has participants sleep for 2, 4 or 6 hours and then has then drive a simulator to measure reaction time to dangerous situations.
  \item A large college wants to know the distribution of race / ethnicity of students applying to the college in 2017, 2018 and 2019.
  \item A local farm tries various planting and irrigation methods to see if it helps with crop growth.
 \end{enumerate}

 \item Astronauts are randomly assigned a set of pills to see if it influences incidence of kidney stones in outer space. They do not know if they are given pills containing potassium citrate or placebos.  They then collect their urine sample while in outer space.
 \begin{enumerate}
 \item Create a diagram that illustrates this experiment.
 \item Identify the subjects of the experiment.
 \item Identify the factors, levels, and treatments in the experiment.
%   \begin{enumerate}
  % \item %Kidney stones in space? (LINK) 
   %   \item NEED ONE HERE.
%  \end{enumerate}
 \end{enumerate}

 \item Decide if the following sampling methods are biased or not. If they are biased, explain why.
 \begin{enumerate}
  \item A news network uses a Twitter survey to report opinions on a current topic.
  \item The survey agency Gallup does a national survey by calling randomly dialed mobile phones.
  \item Portland Tri-Met randomly surveys Red Line trains to collect people's opinions of train services.
  \item PCC does student online evaluations before grades are released at the end of every term to have students give opinions on their classes. 
 \end{enumerate}
\end{enumerate}